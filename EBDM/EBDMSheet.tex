\documentclass[10pt,landscape]{article}
\usepackage{multicol}
\usepackage{calc}
\usepackage{ifthen}
\usepackage[landscape]{geometry}
\usepackage{amsmath,amsthm,amsfonts,amssymb}
\usepackage{color,graphicx,overpic}
\usepackage{tikz}
\usepackage{tikz-3dplot}

\usepackage{array}    % for \newcolumntype
\usepackage{float}    % for [H]
\usepackage{booktabs} % for \toprule, \midrule, \bottomrule



\pdfinfo{
  /Title (EBDM.pdf)
  /Creator (TeX)
  /Producer (pdfTeX 1.40.0)
  /Author (Kevin)
  /Subject (EBDM Cheat Sheet)
  /Keywords (EBDM, tex, latex)}

\geometry{top=.5in,left=.5in,right=.5in,bottom=.5in}

% Turn off header and footer
\pagestyle{empty}

% Redefine section commands to use less space
\makeatletter
\renewcommand{\section}{\@startsection{section}{1}{0mm}%
                                {-1ex plus -.5ex minus -.2ex}%
                                {0.5ex plus .2ex}%
                                {\normalfont\large\bfseries}}
\renewcommand{\subsection}{\@startsection{subsection}{2}{0mm}%
                                {-1explus -.5ex minus -.2ex}%
                                {0.5ex plus .2ex}%
                                {\normalfont\normalsize\textit}}
\renewcommand{\subsubsection}{\@startsection{subsubsection}{3}{0mm}%
                                {-1ex plus -.5ex minus -.2ex}%
                                {1ex plus .2ex}%
                                {\normalfont\small\underline}}
\makeatother

\renewcommand{\baselinestretch}{1.5}

% Don't print section numbers
\setcounter{secnumdepth}{0}


\setlength{\parindent}{0pt}
\setlength{\parskip}{0pt plus 0.5ex}

%My Environments
\newtheorem{example}[section]{Example}
% -----------------------------------------------------------------------

\begin{document}
\raggedright
\footnotesize
\begin{multicols}{3}


% multicol parameters
% These lengths are set only within the two main columns
%\setlength{\columnseprule}{0.25pt}
\setlength{\premulticols}{1pt}
\setlength{\postmulticols}{1pt}
\setlength{\multicolsep}{1pt}
\setlength{\columnsep}{2pt}

\begin{center}
     \Large{\textbf{EBDM}} \\
\end{center}

\section{Definitions}
\textbf{Evidence Based Decision Making:} Clinical decisions by integrating the best available research evidence \newline
\textbf{Epidemiology:} The study of the distribution and determinants of heath and disease in populations \newline
\textbf{Biostatistics:} The application of statistical methods to biological health and medical data \newline
\textbf{Diagnostic Efficacy:} The accuracy and usefulness of a diagnostic test \newline
\textbf{Risk:} Probability of developing the diseases over a specific time period \newline

\subsubsection{Levels of Prevention}
\newcolumntype{P}[1]{>{\centering\arraybackslash}p{#1}}
\noindent \begin{tabular}[H]{|P{1.4cm}|P{2.6cm}|P{2.5cm}|}
\hline
\textbf{Level} & \textbf{Definition} & \textbf{Example}\\
\hline
Primary & Prevent disease before it occurs & Vaccinations, lifestyle counseling \\
\hline
Secondary & Detect disease early to halt or slow it & Pap smear, mammogram \\
\hline
Tertiary & Reduce impact of an established disease & Rehabilitation, physical therapy, chronic disease management \\
\hline
\end{tabular}
\newline

\section{Core Measures}
\subsubsection{Prevalence}
\text{Point} = $\frac{\text{Existing cases at a specific time}}{\text{Population at that time}}$ \newline
\text{Period} = $\frac{\text{Existing cases during a period}}{\text{Population during that period}}$ \newline
\subsubsection{Incidence Rate}
\text{Cumulative} = $\frac{\text{New cases in a time period}}{\text{Population at risk at start of period}}$ \newline
\text{Density} = $ \frac{\text{New cases}}{\sum \text{(time each person is at risk)}}$ \newline

\section{Equations}
\noindent\(\displaystyle
\text{Prevalence} =
\frac{\text{\# subjects with disease}}
     {\text{\# subjects who could have disease}}
\)\newline

\noindent\(\displaystyle
\text{Accuracy} =
\frac{TP + TN}{TP + TN + FP + FN}
\) \newline

\noindent\(\displaystyle
\text{Sensitivity} =
\frac{TP}{TP + FN}
= P(\text{Positive Test} \mid \text{Disease})
\)\newline

\noindent\(\displaystyle
\text{Specificity} =
\frac{TN}{TN + FP}
= P(\text{Negative Test} \mid \text{Healthy})
\)\newline

\noindent\(\displaystyle
\text{False Negative Rate} = 1 - \text{Sensitivity}
\)\newline

\noindent\(\displaystyle
\text{False Positive Rate} = 1 - \text{Specificity}
\)\newline

\subsubsection{Predictive Values}
\textbf{Positive Predictive Value (PPV)} \newline 
\(\displaystyle PPV = P(\text{Disease} \mid \text{Positive Test}) = \frac{TP}{TP + FP}\) \newline
Effect of Prevalence: Increases with higher prevalence. \newline
Effect of Threshold: Raising threshold increases specificity, reduces FP, increases PPV. \newline

\textbf{Negative Predictive Value (NPV)} \newline
\(\displaystyle NPV = P(\text{Healthy} \mid \text{Negative Test}) = \frac{TN}{TN + FN}\) \newline
Effect of Prevalence: Decreases with higher prevalence. \newline
Effect of Threshold: Lowering threshold increases sensitivity, reduces FN, increases NPV. \newline

\textbf{1 - NPV} \newline
\(\displaystyle 1 - NPV = P(\text{Disease} \mid \text{Negative Test}) = \frac{FN}{TN + FN}\) \newline
Interpretation: False-negative probability. \newline

\subsection{Testing Methods}
\subsubsection{SPin vs SNout}
\textbf{SPin:} Specific test to rule in disease \newline
\textbf{SNout:} Sensitive test to rule out disease \newline

\subsubsection{Parallel vs Serial}
\textbf{Serial:} Tests done one after another; paired with SPin \newline
\textbf{Parallel:} Tests done simultaneously; paired with SNout \newline

\subsubsection{Likelihood Ratio}
\textbf{Positive Likelihood Ratio (LR+)} \newline 
\(\displaystyle PPV = P(\text{Disease} \mid \text{Positive Test}) = \frac{TP}{TP + FP}\) \newline
Effect of Prevalence: Increases with higher prevalence. \newline
Effect of Threshold: Raising threshold increases specificity, reduces FP, increases PPV. \newline

\textbf{Negative Predictive Value (NPV)} \newline
\(\displaystyle NPV = P(\text{Healthy} \mid \text{Negative Test}) = \frac{TN}{TN + FN}\) \newline
Effect of Prevalence: Decreases with higher prevalence. \newline
Effect of Threshold: Lowering threshold increases sensitivity, reduces FN, increases NPV. \newline

\section{Odds and Risks}
\textbf{Risk Ratio:} \(\displaystyle \frac{\text{P(Getting Disease} \mid \text{Exposure)}}{\text{P(Getting Disease} \mid \text{No Exposure)}}\) \newline 

\textbf{Odds Ratio:} \(\displaystyle \frac{\text{Odds of (Getting Disease} \mid \text{Exposure)}}{\text{Odds of (Getting Disease} \mid \text{No Exposure)}}\) \newline 

\textbf{Attributable Risk: }\(\displaystyle \text{Risk of (Getting Disease} \mid \text{Exposure)} - \text{Risk of (Getting Disease} \mid \text{No Exposure)}\) \newline

\textbf{Attributable Risk Reduction: }\(\displaystyle \text{Risk of (Getting Disease} \mid \text{Control)} - \text{Risk of (Getting Disease} \mid \text{Treatment)}\) \newline

\textbf{Attributable Risk Proportion: }\(
\displaystyle \frac{R_{\text{exposed}} - R_{\text{unexposed}}}{R_{\text{exposed}}}
\) \newline

\textbf{Number Needed to Treat/Harm: }\(
\displaystyle \frac{1}{\text{AR(R)}} \) \newline

\subsection{Rates}
\textbf{Case Fatality Rate: }\(
\displaystyle \frac{R_{\text{exposed}} - R_{\text{unexposed}}}{R_{\text{exposed}}}
\) \newline

\textbf{Disease Morbidity Rate: }\(
\displaystyle \frac{R_{\text{exposed}} - R_{\text{unexposed}}}{R_{\text{exposed}}}
\) \newline

NNS PSR
Historical Study



\section{Validity}
\subsection{Sampling}
\textbf{Population:} Group of people you are interested in studying \newline
\textbf{Sample:} Subset of the population to collect data from \newline
\textbf{Sample Frame:} List or database the sample is taken from \newline

\subsubsection{Sampling Factors}
\textbf{Procedure:} How the sample was selected \newline
\textbf{Size:} The size of the sample \newline
\textbf{Participation Rate: } How many people of the sample participated \newline

\subsection{Validity}
\textbf{External Validity: } The degree to which your study as being generalized to other situations, peoples, and settings \newline
\textbf{Internal Validity: }The degree to which your study shows a true cause and effect relationship \newline
\textbf{Population Validity: } The degree your sample can be generalized to a larger population \newline
\textbf{Ecological Validity: }The degree you study can be generalized to real-world settings \newline

\section{Statistical Significance}


\end{multicols}
\end{document}\documentclass[10pt,a4paper]{article}
\usepackage[utf8]{inputenc}
\usepackage{amsmath}
\usepackage{amsfonts}
\usepackage{amssymb}
\begin{document}
•
\end{document}